\documentclass[usenames,dvipsnames,aspectratio=169]{beamer}
\usepackage[utf8]{inputenc}
\usecolortheme{seahorse}
\usepackage{enumitem,amsmath,xfrac,dsfont,amsfonts,bbm,cancel,euler,ulem}
\usepackage{apacite,natbib}
\usepackage{hyperref}
\usepackage{caption,subcaption,booktabs,multicol,adjustbox}
\usepackage{xcolor}


\title{The Rise of Market Power and the Macroeconomic Implications  \small{Applied Macroeconomics: Micro Data for Macro Models} }
\author{Author: Jan De Loecker \and Jan Eeckhout \and Gabriel Unger \\ Presented by: Jose M. Quintero}



\AtBeginSection[]
{
  \begin{frame}<beamer>
    \frametitle{Outline}
    \tableofcontents[currentsection]
  \end{frame}
}

\AtBeginSubsection[]
{
   \begin{frame}
        \frametitle{Outline}
        \tableofcontents[currentsubsection]
   \end{frame}
}



\begin{document}

\begin{frame}
  \titlepage
\end{frame}

\begin{frame}{Golden Globe list}
\begin{itemize}[label=\textcolor{teal}{$\blacktriangleright$}]
    \item Estimation of mark-ups using minimal assumptions
    \begin{enumerate}[label=\textbf{\textcolor{teal}{\arabic*.}}]
        \item No stand on market structure or type of competition. 
        \item No assumption on production technology.
        \item Variable inputs are freely adjusted, but capital is not. 
    \end{enumerate}
    \vfill
    \item Results are very compelling and well presented
    \begin{enumerate}[label=\textbf{\textcolor{teal}{\arabic*.}}]
        \item Tackles threads to the message: Overhead costs, profitability, and market value. 
        \item Decomposition of mark-up increase over time. 
    \end{enumerate}
\end{itemize}
\end{frame}

\begin{frame}{Golden Raspberry list}
     \begin{itemize}[label=\textcolor{teal}{$\blacktriangleright$}]
         \item The main identification requires estimating output elasticity
         \begin{enumerate}[label=\textbf{\textcolor{teal}{\arabic*.}}]
             \item Lots of structure assumption ``buried'' in the appendix. 
             \item Elasticity is common across sectors. 
             \item Not clear, within production function methods, how are they estimating elasticity. 
         \end{enumerate}
         \vfill
         \item Sample biases
         \begin{enumerate}[label=\textbf{\textcolor{teal}{\arabic*.}}]
             \item Survival bias (Compustat): Stronger selection on publicly traded companies over time?
             \item Robustness using Economic Census: Elasticity from Compustat + no overhead costs reported. 
         \end{enumerate}
     \end{itemize}
\end{frame}

\begin{frame}{Next Year nominees?} 
    \begin{itemize}[label=\textcolor{teal}{$\blacktriangleright$}]
        \item What are the Macroeconomic implications of market concentration? 
        \begin{enumerate}[label=\textbf{\textcolor{teal}{\arabic*.}}]
            \item Business Dynamism: Akcigit and Ates (2022)
            \item Inequality: Gans et. al (2019). 
        \end{enumerate}
        \item Who is charging higher mark-ups? 
    \end{itemize}
\end{frame}

\begin{frame}[allowframebreaks]{Homework}
Suppose you want to aggregate up microdata to match macro aggregates. 
\begin{enumerate}[label=\textbf{\textcolor{teal}{(\arabic*)}},leftmargin=*]
    \item If you are adding up revenue/COGS across firms, should you weight by firm-level shares of aggregate revenue or firm-level shares of aggregate COGS? 
    \item If you are adding up COGS/revenue across firms, should you weight by firm-level share of aggregate revenue or firm-level shares of aggregate COGS? 
    \vfill
    \item[\textbf{\textcolor{teal}{(S)}}] Aggregate by the share of whatever is in the denominator. To see this let $\mu_i = \sfrac{y_i}{x_i}$ and let the share 
    \begin{equation*}
        s = \frac{x_i}{\sum_{i}x_i}
    \end{equation*}
    Then the aggregate is
    \begin{align*}
        \sum_{i} s_i\mu_i &= \sum_i \frac{x_iy_i}{x_i\sum_nx_n} = \frac{\sum_i y_i }{\sum_i x_i}
    \end{align*}
\end{enumerate}
\break
\begin{enumerate}[resume*]
    \item For counterfactuals such as eliminating markup dispersion or pushing the aggregate markup down to 1, do you want the ``aggregate markup wedge'' to match (Aggregate Revenue)/(Aggregate COGS)?
    \item[\textbf{\textcolor{teal}{(S)}}] The answer is very much dependent on the type of counterfactual it wants to be made. Eliminating dispersion is very different from pushing the aggregate markup to 1. In one case, one might want to match the aggregate wedge, whereas in the other, it might not be the case.  
\end{enumerate}

\end{frame}

\end{document}

