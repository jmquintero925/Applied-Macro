\documentclass[usenames,dvipsnames,aspectratio=169]{beamer}
\usepackage[utf8]{inputenc}
\usecolortheme{seahorse}
\usepackage{enumitem,amsmath,xfrac,dsfont,amsfonts,bbm,cancel,euler,ulem}
\usepackage{apacite,natbib}
\usepackage{hyperref}
\usepackage{caption,subcaption,booktabs,multicol,adjustbox}
\usepackage{xcolor}


\title{Misallocation and Manufacturing TFP in China and India \\ \small{Applied Macroeconomics: Micro Data for Macro Models} }
\author{Author: Chang-Tai Hsieh \and Peter J. Klenow \\ Presented by: Jose M. Quintero}



\AtBeginSection[]
{
  \begin{frame}<beamer>
    \frametitle{Outline}
    \tableofcontents[currentsection]
  \end{frame}
}

\AtBeginSubsection[]
{
   \begin{frame}
        \frametitle{Outline}
        \tableofcontents[currentsubsection]
   \end{frame}
}



\begin{document}

\begin{frame}
  \titlepage
\end{frame}

\begin{frame}{Golden Globe list}
\begin{itemize}[label=\textcolor{teal}{$\blacktriangleright$}]
    \item Tracktable model 
    \begin{enumerate}[label=\textbf{\textcolor{teal}{\arabic*.}}]
        \item Easy interpretation of the results.
        \item Very clear about the underlying assumptions. 
    \end{enumerate}
    \vfill
    \item Tackles a very relevant question 
    \begin{enumerate}[label=\textbf{\textcolor{teal}{\arabic*.}}]
        \item Opens a black box: What generates differences in TFP across countries?
        \item Misallocation of resources arises endogenously. 
    \end{enumerate}
\end{itemize}
\end{frame}

\begin{frame}{Golden Raspberry list}
     \begin{itemize}[label=\textcolor{teal}{$\blacktriangleright$}]
         \item The estimates are very big
         \begin{enumerate}[label=\textbf{\textcolor{teal}{\arabic*.}}]
             \item TFP growth: China - 30\%-50\% and India - 40\%-60\%. 
             \item What's the implied GDP growth?
         \end{enumerate}
         \vfill
         \item Variable mark-ups
         \begin{enumerate}[label=\textbf{\textcolor{teal}{\arabic*.}}]
             \item The argument for not considering mark-ups falls short. 
             \item Model variable mark-ups and non-linear demand. 
         \end{enumerate}
     \end{itemize}
\end{frame}

\begin{frame}{Next Year nominees?} 
    \begin{itemize}[label=\textcolor{teal}{$\blacktriangleright$}]
        \item Flexibility:
        \begin{enumerate}[label=\textbf{\textcolor{teal}{\arabic*.}}]
            \item Variable Mark-ups: Melitz and Ottaviano (2008). 
            \item Network structure: Baqaee and Farhi (2020).
        \end{enumerate}
        \vfill
        \item Bring the dynamics. All the misallocations are static. 
        \begin{enumerate}[label=\textbf{\textcolor{teal}{\arabic*.}}]
            \item Misallocation of resources is associated with firm dynamics. See Acemoglu et. al (2018). 
            \item Firm entry? Exit? Over investment in R\&D? 
        \end{enumerate}
        \vfill
        \item Next question: How are these wedges arising? Market conditions
        \begin{enumerate}[label=\textbf{\textcolor{teal}{\arabic*.}}]
            \item Market conditions: Informality (Ulyssea, 2018). 
            \item Credit Constrains (Banerjee and Dufflo, 2005)
        \end{enumerate}
    \end{itemize}
\end{frame}

\begin{frame}[allowframebreaks]{Homework}
\begin{enumerate}[label=\textbf{\textcolor{teal}{(\alph*)}},leftmargin=*]
    \item Marginal tax rate differences: Yes. 
    \item Price markup dispersion: No. Kimball preferences generates mark-up dispersion without a distortion. 
    \item Wage markdown dispersion: No.
    \item A subset of firms are SOE that do not maximize their profits: Yes
    \item Size-dependent regulations (e.g. firing costs for big firms): Yes. 
    \item Adjustment costs: Not necessarily.
    \item Transportation costs: No. See Melitz (2003) where there are iceberg cost but the economy is efficient. 
    \item Overhead costs: No. See Melitz (2003) where there are fix cost but the economy is efficient. 
\end{enumerate}

\end{frame}

\end{document}

